\documentclass{article}

% Language setting
% Replace `english' with e.g. `spanish' to change the document language
\usepackage[english]{babel}

% Set page size and margins
% Replace `letterpaper' with `a4paper' for UK/EU standard size
\usepackage[letterpaper,top=2cm,bottom=2cm,left=3cm,right=3cm,marginparwidth=1.75cm]{geometry}

% Useful packages
\usepackage{amsmath}
\usepackage{ifluatex}
\ifluatex
  \usepackage{pdftexcmds}
  \makeatletter
  \let\pdfstrcmp\pdf@strcmp
  \let\pdffilemoddate\pdf@filemoddate
  \makeatother
\fi
\usepackage{svg}
\svgsetup{
    inkscape=pdf,
    inkscapeexe={/usr/bin/inkscape -z -C}
}
\usepackage{graphicx}
\usepackage[colorlinks=true, allcolors=blue]{hyperref}

\title{Biometrics: Fingerprint recognition}
\author{Yauheni Zviazdou}

\begin{document}
\maketitle


\section{Introduction}

I chose fingerprint \(id = 3; num = 7;\) (I'm 11th in the list, but database has just 10 fingerprints series).

\section{Preprocessing}

\subsection{Fingerprint processing without segmentation}

Usually people can process fingerprints without prepossessing, but recognition with computer methods use methods based on computing orientation and frequency.

\begin{figure}[htbp] 
  \centering
  \def\svgscale{0.5}
  \includesvg{NoSegmentation/NoSegmentationOrientation.svg}
  \caption{Example of computing orientation field without segmentation}
  \label{NoSegmentationOrientation}
\end{figure}

\begin{figure}[htbp]
  \centering
  \def\svgscale{0.5}
  \includesvg{NoSegmentation/NoSegmentationFrequency.svg}
  \caption{Computing a frequency of non-segmented fingerprint image.}
\end{figure}

\newpage

\subsection{Fingerprint segmentation from background}
For segmentation fingerprint from background I chose method based on  image pixel intensity variance thresholding. I compute variance of blocks \(blockSize\times blockSize\) and threshhold it. By default I use \textit{blockSize=16} and \(threshold=0.001\). Also I implemented method that removes misclassified foreground blocks (by default \textit{minBlockSize = blockSize * 5} ).
\begin{figure}[htbp]
  \centering
  \def\svgscale{0.6}
  \includesvg{Segmentation/Segmentation.svg}
  \caption{Fingerprint segmentation from background.}
\end{figure}
\begin{figure}[htbp]
  \centering
  \def\svgscale{0.6}
  \includesvg{Segmentation/SegmentationOrientation.svg}
  \caption{Visualization of orientation field}
\end{figure}
\begin{figure}[htbp]
  \centering
  \def\svgscale{0.4}
  \includesvg{Segmentation/SegmentationFrequency.svg}
  \caption{Visualization of frequency}
\end{figure}
\begin{figure}[htbp]
  \centering
  \def\svgscale{0.6}
  \includesvg{Segmentation/SegmentationGabor.svg}
  \caption{Enhancement using Gabor filters}
\end{figure}
\begin{figure}[htbp]
  \centering
  \def\svgscale{0.6}
  \includesvg{Segmentation/SegmentationMinutia.svg}
  \caption{Highlighting minutiae.}
\end{figure}

\newpage

\subsection{Different types of fingerprint scanners}

There are some types of scanners: \textbf{Optical scanners},\textbf{Capacitive or CMOS scanners},\textbf{Ultrasonic fingerprint scanners},\textbf{Thermal scanners}.

There are examples of preprocessed images of my fingerprint using different scanners:
\begin{figure}[htbp] 
  \def\svgscale{0.5}
  \includesvg{CustomFingerprints/f_1_13_SegmentationMinutia.svg}
  \def\svgscale{0.5}
  \includesvg{CustomFingerprints/f_1_14_SegmentationMinutia.svg}
  \def\svgscale{0.5}
  \includesvg{CustomFingerprints/f_1_15_SegmentationMinutia.svg}
  \def\svgscale{0.5}
  \includesvg{CustomFingerprints/f_1_17_SegmentationMinutia.svg}
  \caption{Examples of preprocessed fingerprints from different scanners}
  \label{scanners}
\end{figure}


As you can see at Figure~\ref{scanners} majority of minutiae are highlighted right, but for better performances it's good to use different segmentation methods (or different parameters \textit{threshhold} and \textit{blockSize}) for different scanners.

\newpage
\section{Fingerprint Matching}
\subsection{Example Matching (same person, same scanner)}

\begin{figure}[htbp]
  \centering
  \def\svgscale{0.3}
  \includesvg{ExampleMatching/Example.svg}
  \caption{Two fingerprints from same person.}
\end{figure}

\begin{figure}[htbp]
  \centering
  \def\svgscale{0.5}
  \includesvg{ExampleMatching/ExampleAlign.svg}
  \caption{Aligning the fingerprint}
\end{figure}

\begin{figure}[htbp]
  \def\svgscale{0.5}
  \includesvg{ExampleMatching/ExampleMinutia1.svg}
  \def\svgscale{0.5}
  \includesvg{ExampleMatching/ExampleMinutia2.svg}
  \caption{Highlighted minutiae in two fingerprints.}
\end{figure}

\newpage
number of matched minutiae : 13

number of minutiae in input image : 22

number of minutiae in database image : 23
 
Score ot the minutiae matching: 0.46984

Score ot the fingercode matching: 0.00022896

13/22 matched minutiae with just 0.42 minutiae score is a good result!

\subsection{Attempt to find best matching}

There are two fingerprints with best matching from \(fvc02\_4\):
\begin{itemize}
  \item id = 3, num = 7
  \item id = 1, num = 2
\end{itemize}


\begin{figure}[htbp]
  \centering
  \def\svgscale{0.3}
  \includesvg{BestMatch/BestMatch.svg}
  \caption{Two fingerprints from different persons.}
\end{figure}
\newpage

\begin{figure}[htbp]
  \centering
  \def\svgscale{0.5}
  \includesvg{BestMatch/BestMatchAlign.svg}
  \caption{Aligning the fingerprint}
\end{figure}

\begin{figure}[htbp]
  \def\svgscale{0.5}
  \includesvg{BestMatch/BestMatchMinutia1.svg}
  \def\svgscale{0.5}
  \includesvg{BestMatch/BestMatchMinutia2.svg}
  \caption{Highlighted minutiae in two fingerprints.}
\end{figure}

number of matched minutiae : 4

number of minutiae in input image : 20

number of minutiae in database image : 23

Score ot the minutiae matching: 0.5548

Score ot the fingercode matching: 0.004335

\newpage
\subsection{Matching results of same person, but with different scanners}

\begin{figure}[htbp]
  \centering
  \def\svgscale{0.3}
  \includesvg{DifferentScanners/Different.svg}
  \caption{Two fingerprints from same persons.}
\end{figure}

\begin{figure}[htbp]
  \def\svgscale{0.5}
  \includesvg{DifferentScanners/DifferentMinutia1.svg}
  \def\svgscale{0.5}
  \includesvg{DifferentScanners/DifferentMinutia2.svg}
  \caption{Highlighted minutiae in two fingerprints.}
\end{figure}

number of matched minutiae : 0

number of minutiae in input image : 9

number of minutiae in database image : 25

Score for minutiae : NaN

Score ot the minutiae matching: NaN

Score ot the fingercode matching: 0.0056104

\textbf{Conclusion}: We can't use different scanners to identify person. There is very big risk of identification error.

\end{document}